\documentclass[letterpaper,12pt]{article}
\usepackage{xcolor}
\usepackage{latexsym}
\usepackage[empty]{fullpage}
\usepackage[top = 40pt, left = 1in, right = 1in, headheight = 1cm, headsep = 15pt]{geometry}
\usepackage{titlesec}
\titlelabel{\thetitle.\quad}
\usepackage{marvosym}
\usepackage[usenames,dvipsnames]{color}
\usepackage{verbatim}
\usepackage{enumitem}
\usepackage[hidelinks]{hyperref}
\usepackage{fancyhdr}
\usepackage{calc}
\usepackage[english]{babel}
\usepackage{tabularx}
\usepackage{kpfonts}
\usepackage[T1]{fontenc}
\input{glyphtounicode}
\usepackage{graphicx}
\graphicspath{{Images/}}

%Redefine fancy page style to get rid of footer, header
\fancypagestyle{plain}{%
\fancyhf{}% clear all header and footer fields
}
\fancyhf{}% Clear all headers/footers
\fancyhead[L]{\textbf{\textit{Software Requirements Specification for HMS}}}
\fancyhead[R]{\textbf{\textit{Page \thepage}}}
\thispagestyle{plain}
\pagestyle{fancy}
\pagenumbering{roman}

\begin{document}
\topskip0pt
\vspace*{\fill}
\begin{flushright}
    \textbf{
        \Huge{Software Requirements \\ Specification} \\ \vspace{30pt} 
        \LARGE{for} \\ \vspace{30pt}
        \Huge{Hall Management System} \\ \vspace{30pt}
        \Large{Version 1.0} \\ \vspace{30pt}
        \Large{Prepared by}
    } \\ \vspace{10pt}
    \large{Tanishq Prasad \\ Sourodeep Datta \\ Krish Khimasia} \\ \vspace{30pt}
    \textbf{
        \Large{Sorrow, Indian Institute of Technology Kharagpur} \\ \vspace{30pt}
        \large{21/03/2023}
    }
\end{flushright}
\vspace*{\fill}
\pagebreak

\renewcommand*\contentsname{\Huge Table of Contents}
\small\tableofcontents
\pagebreak

\section*{\LARGE Revision History}
    \begin{center}
        \begin{tabular}{| p{2cm} | p{2cm} | p{5cm} | p{2cm} |}
            \hline			
            \textbf{Name} & \textbf{Date} & \textbf{Reason For Changes} & \textbf{Version} \\
            \hline \hline
            A & B & C & D \\
            \hline
            E & F & G & H \\
            \hline
        \end{tabular}
    \end{center}
\begin{figure}
    \centering
    \includegraphics[width = 10cm]{im.jpg}
\end{figure}
\section{\LARGE Introduction}
\subsection{\Large Purpose}
    The purpose of this document is to present a detailed description of the Hall Management System. It will explain the purpose and features of the system, the interfaces
of the system, what the system will do, the constraints under which it must operate and
how the system will react to external stimuli. This document is intended for both the
stakeholders and the developers of the system and will be proposed to the  Management Centre (HMC) for its approval.
\subsection{\Large Product Scope}
This software system will be a Hall Management System for our institute. This system will be designed to streamline the process of everything related to the working of halls in our institute. It will serve as a portal for the students, wardens, hall clerks and the HMC Chairman to access information relevant to the respective parties.

More specifically, this system is designed to facilitate communication between the different entities such as students and wardens, which would ensure smoother running of the halls and provide an organised way to handle expenses and allocations.

\subsection{\Large References}
IEEE. \emph{IEEE Std 830-1998 IEEE Recommended Practice for Software Requirements
Specifications.} IEEE Computer Society, 1998.

\textcolor{red}{also add the sample provided as reference?}
\subsection{\Large Overview of Document}
The next chapter, the Overall Description section, of this document gives an
overview of the functionality of the product. It describes the informal requirements and is
used to establish a context for the technical requirements specification in the next chapter.

The third chapter, Requirements Specification section, of this document is written
primarily for the developers and describes in technical terms the details of the
functionality of the product.

Both sections of the document describe the same software product in its entirety,
but are intended for different audiences and thus use different language.
\section{\LARGE Overall Description}
\subsection{\Large Use Cases (Don't know if right place}
\subsubsection{Student Use Cases}
\begin{itemize}
    \item \textbf{Hall Assignment & Room Allotment}
    \item \textbf{Get Fees}
    \item \textbf{File a complaint}
\end{itemize}
\subsubsection{Hall Clerk Use Cases}
\begin{itemize}
    \item \textbf{}
    \item \textbf{}
    \item \textbf{}
\end{itemize}
\subsubsection{HMC Chairman Use Cases}
\begin{itemize}
    \item \textbf{Allocate Funds}
    \item \textbf{View Overall Occupancy}
\end{itemize}
\subsubsection{Hall Warden Use Cases}
\begin{itemize}
    \item \textbf{}
    \item \textbf{}
    \item \textbf{}
\end{itemize}
\subsubsection{Mess Manager Use Cases}
\begin{itemize}
    \item \textbf{}
    \item \textbf{}
    \item \textbf{}
\end{itemize}
\subsection{\Large Product Functions}
\begin{itemize}
    \item Students can get a breakdown of their fees and pay them.
    \item Students can raise various types of complaints to the Wardens.
    \item Hall Clerks can enter any leave taken by attendants or gardeners as well add petty charges to the hall budget.
    \item Mess Managers can generate can generate a report of mess dues.
    \item Wardens can generate of report of the hall finances.
    \item HMC Chairman can distribute the annual grant among different halls and view overall room occupancy.
    \item Hall Wardens can view their respective hall's room occupancy, enter expenditure details and view and act on complaints raised by the students.
    \item Mess Managers will handle the mess accounts of each student of the hall.
\end{itemize}
\subsection{\Large User Classes and Characteristics}
\begin{itemize}
    \item The Student is expected to be Internet literate
    \item The Hall Clerk is expected to be Internet literate and have basic knowledge of mathematics
    \item The HMC Chairman is expected to be Internet literate and have basic knowledge of finance and mathematics
    \item The Hall Warden is expected to be Internet literate and have basic knowledge of finance and mathematics
    \item The Mess Manager is expected to be Internet literate and have basic knowledge mathematics
    
\end{itemize}
\subsection{\Large Operating Environment}
The software will operate on a Linux system, mainly consisting of a python program, running Django with an SQLite database. It will host a website, which will be accessible to the users.
\subsection{\Large System Environment}
The Hall Management System has five active actors and one cooperating system. Every user can access this system directly via the website

\end{document}
